\documentclass[a4paper,12pt,twocolumn]{article}
\usepackage{indentfirst}
\setlength{\parindent}{2em}
\usepackage[colorlinks = true]{hyperref}
\usepackage{cite}
\usepackage{graphicx}
\usepackage{booktabs}
\usepackage{subfig}
\usepackage{amsmath,amsfonts,amssymb}

\title{Generative Adversarial Network}

\author{Yuan An}

\begin{document}
\maketitle
In the last article, it was introduced that researchers made GAN-based AI create new levels for Doom and Super Mario games. Today, I read something about GAN.
\par
Generative adversarial networks are a class of artificial intelligence algorithms used in unsupervised machine learning \cite{GANwikipedia}. This technique can generate photographs that look at least superficially authentic to human observers, having many realistic characteristics.
\par
This concept was introduced by IJ Godfellow \emph{et al.} in 2014 \cite{Goodfellow2014Generative}. GANs have been used to produced samples of photorealistic\footnote{a genre of art that encompasses painting, drawing and other graphic media.} images for the purposes of visualizing new interior/industrial design or items for computer games' scenes.
\par
\begin{figure}
	\centering
	\includegraphics[width=0.9\linewidth]{GAN.jpg}
	\caption{Adversarial Nets Framework}\label{ANF}
\end{figure}
\par
In Godfellow's paper, an equation (Equ.~\ref{equ1}) is given to be described in mathematical linguistics. This equation is difficult to understand.
\begin{equation}
\begin{split}
	&\min_{G} \max_{D}V(D,G) = \mathbb{E}_{\boldsymbol{x}\thicksim p_{data}(\boldsymbol{x})}[\log{D(\boldsymbol{x})}]\\
	&+\mathbb{E}_{\boldsymbol{z}\thicksim p_{\boldsymbol{z}} (\boldsymbol{z})} [\log{(1-D(G(\boldsymbol{z})))}]
\end{split}
\label{equ1}
\end{equation}
\par
As shown in Fig.~\ref{ANF}, the figure is easy to understand. The left part displays when a sample $x$ from data is input to the function $D$ which is called Discriminator, and tries to Discriminator make $D(x)$ be near 1 as $x$ is from real data. The right part shows that function $G$ which is called Generator generates a fake data $G(z)$ by accepting a noise $z$, then this fake data is input to function $D$, Discriminator tires to make $D(G(z))$ near 0 as $G(z)$ is not a real data, while Generator tires to make $D(G(z))$ near 1 for its inherent function.
\par
That is to say, Discriminator tries its best to distinguish the real data from the fake data , and Generator also does its best to generate a 'real' data to cheat Discriminator.
\bibliography{reference}

\bibliographystyle{plain}

\end{document}

