\documentclass[a4paper,12pt]{article}
\usepackage{graphicx}
%opening
\title{YouTube publishes deleted videos report}
\author{Yuan An}

\begin{document}
\maketitle

YouTube's first three-monthly "enforcement report" reveals the website deleted 8.3 million videos between October and December 2017 for breaching its community guidelines.
\par
\begin{figure}[htp]
	\centering
	\includegraphics[scale=0.4]{picture1.png}
	\caption{Who or what first reported YouTube's struck videos?}\label{pic1}
	\centering
	\includegraphics[scale=0.4]{picture2.png}
	\caption{Human-reported flags sent to YouTube}\label{pic2}
\end{figure}
\par
As is shown in the Figure~\ref{pic1}, YouTube said tis algorithms had 6.7 million videos that had then been sent to human moderators and deleted. That is to say, only a few of videos are reported by YouTube's users.
\par
In the Figure~\ref{pic2}, Sexual is the most flags reported by users. And not all flagged videos were removed, and some videos were labeled for more than one reason. 
\par
From my point of view, video websites should strength content supervision for sites' visitors range from children to elderly men. Kids lack the ability to distinguish right from wrong. Sites should delegate more power to users, and make it available for the trusted individual to manage contents uploaded by other users.
\end{document}
