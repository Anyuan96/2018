\documentclass[a4paper,12pt]{article}
\usepackage[colorlinks = true]{hyperref}
\usepackage{cite}
\usepackage{graphicx}
\usepackage{amsmath}
\usepackage{booktabs}
%opening
\title{Learning artificial neural network}
\author{Yuan An}

\begin{document}

\maketitle
Artificial neural networks\cite{ANNwikipedia} (ANNs) are computing systems vaguely inspired by the biological neural networks that constitution animal brains. An ANN is based on a collection of connected units or nodes named artificial neurons. Each connection between artificial neurons can transmit a signal from one to another. 
\begin{figure}[h]
	\centering
	\includegraphics[scale=0.5]{ANN.png}
	\caption{The model of ANN}\label{ANNmodel}
\end{figure}
\par
As shown in the figure above, each circle represents an artificial neuron as an arrow represents ra connection from the output of one artificial neuron to the input of another.
\par
In Figure~\ref{ANNmodel}, we can see some visible components of an ANN: neurons and connections. There are other components behind representation.
\par
Neurons: A neuron with label $j$ receiving an input $p_j(t)$ from front neurons of the following components: an activation $a_j(t)$ depending on a discrete time parameter, a threshold $\theta_j$ staying fixed unless changed by a learning function, an activation function $f$ that computes the new activation at a given time $t+1$ from $a_j(t)$, $\theta_j$, and the net input $p_j(t)$ giving rise to the relation $a_j(t+1)=f(a_j(t),p_j(t),\theta_j)$ and an output function $f_{out}$ computing the output from the activation $o_j(t)=f_{out}(a_j(t))$. 
\par
\begin{table}[h]
	\centering
	\renewcommand{\arraystretch}{1.1}
	\begin{tabular}{cc}
		\toprule
		Components&Expression\\
		\midrule
		activation&$a_j(t)$\\
		threshold&$\theta_j$\\
		activation function& $a_j(t+1)=f(a_j(t),p_j(t),\theta_j)$\\
		output function& $o_j(t)=f_{out}(a_j(t))$\\
		weight&$w_{ij}$\\
		propagation function&$p_j(t) = \sum\limits_{i}o_i(t)w_{ij}$\\
		\bottomrule
	\end{tabular}
	\caption{Components of ANN}
\end{table}
Connections: each connection transferring the output of a neuron $i$ to the input of a neuron $j$, and each connection is assigned a weight $w_{ij}$.
\par
Propagation function: the propagation function computes the input $p_j(t)$ to the neuron $j$ from the outputs $o_i(t)$ of front neurons and typically has the form $p_j(t) = \sum\limits_{i}o_i(t)w_{ij}$.
\par
Learning rule: the learning rule is a rule or an algorithm which modifies the parameters of the neural network, in order for a given input to the network to produce a favored output. This learning process typically amounts to modifying the weights and thresholds of the variables within the network.
\bibliographystyle{plain}
\bibliography{reference}

\end{document}
