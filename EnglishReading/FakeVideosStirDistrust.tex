\documentclass[a4paper, 12pt]{article}
\usepackage{graphicx}
\usepackage[colorlinks = true]{hyperref}
%opening
\title{Fake videos stir distrust}
\author{An Yuan}

\begin{document}

\maketitle

There was a \href{https://www.buzzfeed.com/davidmack/obama-fake-news-jordan-peele-psa-video-buzzfeed?utm_term=.onGLbwdPQ#.ml29GBDn8}{video} featuring Barack Obama about a character from the hit movie \emph{Black Panther}, and then making a crude remark about Donald Trump, which raise new concerns about this subject. It was not of course genuine. It was the actor Jordan Peele's voice had been synced to the lips of the former US president.
\begin{figure}[h]
	\centering
	\includegraphics[scale=0.5]{BarackObama.jpg}\\
	Fig 1.~At the TED conference, Google engineer Supasorn Suwajanakorn demonstrated the AI-faked video
\end{figure}
\par
The video was created by Buzzfeed using a software called FakeAPP. That sounds a bit scary. Will we stop believing video and audio what we looked and listened? It's a challenge to the seeing is believing idea.
\par
In my opinion, we should be more skeptical about the sources of information, strengthen our ability of discrimination, and be more vigilant about what is real and what is fake. 
\thispagestyle{empty}
\end{document}
