\documentclass[a4paper, 12pt]{article}
\usepackage{graphicx}
%opening
\title{Are AI fairytales the future?}
\author{Yuan An}

\begin{document}

\maketitle

It was recently reported that the meditation app Calm had published a ``new''  fairytale by the Brothers Grimm.
\par
This story is named \emph{The Princess and the Fox}, and collaborated with Botnik\footnote{a community of writers, artists and developers.}. In fact, the words and phrases of the fairytale are generated by program, and are pieced together sentences to story by human writers. 
\par
Though it doesn't produce the best prose, Botnik still provides predictive keyboard online for users to experiment.

\begin{center}
	\includegraphics[scale=0.4]{CreativeKeyboard.jpg}\\
	Figure 1.~Creative Keyboard on Botnik.org
\end{center}

In my opinion, it is progress in Natural Language Processing(NLP) and  has positive effects on the application of NLP. We can using NLP in machine translation, data mining public opinion supervision and so on. It must  bring convenience for future life in information age.
\end{document}
